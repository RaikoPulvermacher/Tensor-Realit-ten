\documentclass[12pt,a4paper]{article}

\usepackage[utf8]{inputenc}
\usepackage[T1]{fontenc}
\usepackage[english]{babel}
\usepackage{geometry}
\usepackage{graphicx}
\usepackage{hyperref}
\usepackage{setspace}
\usepackage{parskip}
\usepackage{microtype}
\usepackage{amsmath}
\usepackage{fancyhdr}
\usepackage{titlesec}

\geometry{a4paper, margin=2.5cm}

\hypersetup{
  colorlinks=true,
  urlcolor=blue,
  linkcolor=black,
  pdftitle={Pulvermacher -- Foundation of Nature},
  pdfauthor={Raiko Pulvermacher},
  pdfsubject={A Bottom-Up Description of Reality},
  pdfkeywords={Tensor, Reality, Quantum Mechanics, Bottom-Up, Nature, Superposition}
}

\pagestyle{fancy}
\fancyhf{}
\rhead{Raiko Pulvermacher}
\lhead{Foundation of Nature}
\cfoot{\thepage}

\title{
  \vspace{-1cm}
  \Huge\bfseries Pulvermacher -- Foundation of Nature\\[0.4em]
  \large A Bottom-Up Description of Reality
}
\author{
  Raiko Pulvermacher\\[0.3em]
  \small\href{https://orcid.org/0009-0003-9431-1001}{ORCID: 0009-0003-9431-1001}\\
  \small\href{https://osf.io/py42t/}{OSF: https://osf.io/py42t/}\\
  \small E-Mail: \href{mailto:Pulvermacher.Raiko@web.de}{Pulvermacher.Raiko@web.de}
}
\date{December 2025 -- February 2026}

\begin{document}

\maketitle
\thispagestyle{empty}

\begin{abstract}
\noindent
This work presents a structural bottom-up view of nature by Raiko Pulvermacher.
It is not a physical model, not a formal theory, and not a competitor to existing
physical theories. It is a structural description of how nature appears when it is
not separated, but treated as a coherent whole. The entirety of existence is based
on an inseparable coupling of six fundamental aspects, described as the tensor
$C_{uv}(E, T, I, Z, G, M)$: Energy~(E), Tunnel~Effect~(T), Information~(I),
Time~(Z), Gravitation~(G) and Matter~(M).
\end{abstract}

\tableofcontents
\newpage

%% ============================================================
\section{Main Text: The Completeness of Nature}
%% ============================================================

When nature is not explained from top to bottom -- that is, not through
ready-made theories, measurement models, or isolated formulas -- but is
instead observed from bottom to top, a different picture emerges. Nature
then no longer appears as a collection of separate laws, but as a single,
coherent event that is self-similar on all levels.

There is no preferred starting point. There is only existence -- and
existence always occurs completely.

At the origin there is no particle, no force, and no time, but an
inseparable structure of three aspects:

\begin{center}
\textbf{Energy -- Tunnel Effect -- Information \quad (E -- T -- I)}
\end{center}

This triangle is not symbolic, but real. None of the three aspects can
exist on its own. Energy without transition remains trapped. Tunnelling
without energy is meaningless. Information without energy and transition
does not exist.

The tunnel effect is not a special phenomenon but the fundamental mechanism
of every change. Every movement, every reaction, every fusion, and every
transition means that a barrier is overcome. Without tunnelling there would
be no movement, no temperature, no fusion, no stars, no elements, and no
matter.

From this quantum foundation a second, equally important structure
necessarily unfolds:

\begin{center}
\textbf{Time -- Gravitation -- Matter \quad (Z -- G -- M)}
\end{center}

This structure is not a new level, but the visible consequence of the lower
one. Matter is condensed energy. Gravitation is the effect of this
condensation. Time is the measure of change under condensation. Time does
not arise independently, but only where matter exists and gravitation acts.
It follows necessarily: no matter means no gravitation, no time, and no
space.

Both structures are not separate systems. They are two perspectives on the
same reality. The lower structure describes the non-local, superposed
existence; the upper one describes the locally perceivable reality. The
transition between the two does not happen abruptly, but scales.

This scaling is not linear, but recursive. It appears in nature as a
Fibonacci structure -- not as decoration, but as an expression of natural
growth and condensation processes. Whether atom, star, planet, galaxy or
black hole: the pattern remains the same; only the pulsation changes.

Nature is not a puzzle of separate pieces, but a coherent tensor of states.
Superposition is not merely a probability concept, but the real coexistence
of possible states, each separated by different proper times. The connection
of these states can be expressed as a coupling:

\begin{center}
\Large$\text{Reality} = C_{uv}(E,\, T,\, I,\, Z,\, G,\, M)$
\end{center}

This tensor does not describe the collapse of states, but their simultaneous
existence.

Condensation creates gravitation, gravitation changes time, and time
determines the observable structure. The stronger the condensation, the
tighter the temporal pulsation. Extremely condensed systems appear timeless
or hidden from the outside. Black holes are therefore not objects outside
of nature, but states of maximum condensation in which the internal time is
so strongly compressed that it can no longer be resolved from the outside.

Nature does not exist in parts. It exists only completely. Everything
generates everything, everything acts on everything, everything is
synchronous. Top-down models capture fragments, while bottom-up makes the
structure of the whole visible.

This work presents neither a physical theory nor a formal model and is not
in competition with existing theories. It is a structural, nature-related
description showing how reality presents itself when its aspects are not
separated. Classical top-down theories are not negated, but understood as
special cases of measurement.

%% ============================================================
\section{Diagrams}
%% ============================================================

The following diagrams complement the text and illustrate the structural
relationships. Each figure illuminates a different aspect of the Tensor
of Realities.

\begin{figure}[h!]
  \centering
  \includegraphics[width=\textwidth]{Tensor_of_Realities_en}
  \caption{Overall representation of the Tensor of Realities
           $C_{uv}(E, T, I, Z, G, M)$}
  \label{fig:tensor}
\end{figure}

\begin{figure}[h!]
  \centering
  \includegraphics[width=\textwidth]{Superposition_en}
  \caption{Superposition: Real coexistence of possible states}
  \label{fig:superposition}
\end{figure}

\begin{figure}[h!]
  \centering
  \includegraphics[width=\textwidth]{Matter_en}
  \caption{Matter as condensed energy}
  \label{fig:matter}
\end{figure}

\begin{figure}[h!]
  \centering
  \includegraphics[width=\textwidth]{Gravitation_en}
  \caption{Gravitation as the effect of condensation}
  \label{fig:gravitation}
\end{figure}

\begin{figure}[h!]
  \centering
  \includegraphics[width=\textwidth]{Time_en}
  \caption{Time as the pulsation of the spatial structure}
  \label{fig:time}
\end{figure}

\begin{figure}[h!]
  \centering
  \includegraphics[width=\textwidth]{Energy_Escape_en}
  \caption{Energy Escape}
  \label{fig:energy}
\end{figure}

\begin{figure}[h!]
  \centering
  \includegraphics[width=\textwidth]{Atomic_Structure_en}
  \caption{Atomic Structure in the Tensor model}
  \label{fig:atomic}
\end{figure}

\begin{figure}[h!]
  \centering
  \includegraphics[width=\textwidth]{Neutron_Development_en}
  \caption{Neutron Development}
  \label{fig:neutron}
\end{figure}

\clearpage

%% ============================================================
\section{Methodology: The Pulvermacher Foundation}
%% ============================================================

\subsection{Introduction and Methodology}

This model is the result of an intensive, independent logical derivation
developed within four weeks -- starting from an engagement with quantum
superposition. It breaks radically with the classical ``top-down'' view
and replaces it with a ``bottom-up'' principle. The foundation assumes
that nature knows no separate laws, but forms an inseparable unity.

\subsection{The Tensor of Realities}

The entirety of existence is based on an inseparable coupling of six
fundamental aspects, described as the tensor $C_{uv}(E, T, I, Z, G, M)$.
These divide into two operating triangles:

\begin{itemize}
  \item \textbf{The operational foundation:} Energy~(E), Tunnel~Effect~(T)
        and Information~(I).
  \item \textbf{The form of appearance:} Time~(Z), Gravitation~(G) and
        Matter~(M).
\end{itemize}

No aspect can exist in isolation or be changed without immediately
influencing the others. Matter~(M) is nothing other than the maximum
condensation of space, while Gravitation~(G) is the resulting pressure
gradient.

\subsection{Time as Pulsation of Information}

In this model, Time~(Z) is not a linear dimension, but the oscillation or
pulsation of the spatial structure. The higher the matter density
(condensation), the tighter and faster the local pulse. Light (photons)
functions as an information carrier~(I) that ``freezes'' the pulse of its
origin and transports it. The ``age'' of a photon is therefore not a
measure of time, but a measure of the stored pulse state.

\subsection{Proof: The Light-Pulse Thought Experiment}

The primacy of information over geometric distance is demonstrated by the
following experiment: a photon releases its quantum information at the
moment of its creation (e.g., ``I am 5 metres old'', according to its
pulse state). If this photon were to skip part of its path through
tunnelling~(T) or teleportation, it retains this information. If it
reaches an observer's retina after a physical 15 metres, it still shows
the pulse of 5 metres. We therefore do not perceive space, but the
pulse information stored in light.

\subsection{The Gravitational Lens as Pulse Synchronisation}

The gravitational lensing effect confirms this principle: a large
mass~(M) condenses space and thus also the local pulse~(Z). When light
passes through this region, it must adapt its amplitude to the higher
pulse frequency. The resulting change in direction is not a deflection
by an external force, but the necessary synchronisation of energy~(E) to
the condensed spatial structure. This simultaneously explains red and
blue shift: it is the expression of how light must adapt its information
when moving between different condensation states (pulses).

\subsection{Resolution of the Perspective Illusion}

Everyday misperceptions -- such as the assumption of an absolutely
straight propagation of light -- are resolved by the bottom-up principle.
The observer is themselves part of the local pulsation. What appears
``straight'' to them is, in the global reality of the condensed tensor,
a curve. We do not perceive geometric truth, but the interpretation of
the arriving information pulses.

\subsection{Conclusion}

Modern quantum mechanics, which extracts information from photons,
unwittingly already confirms this model. The Pulvermacher Foundation
provides the necessary bridge to understand the quantum world~(E-T-I) and
the macroscopic world~(Z-G-M) as a single, coherent system of space
condensation. Nature is a pulse -- and information is its witness.

%% ============================================================
\section{Classification Relative to Existing Theories}
%% ============================================================

Top-down theories (e.g., Copenhagen interpretation, theory of relativity,
standard models) are not negated here and not devalued. They are
understood as special cases of measurement -- that is, as descriptions of
\emph{what is observed}, not of \emph{what is}.

This document does not describe a measurement prescription, but a coherent
view of natural interconnections.

%% ============================================================
\section{License}
%% ============================================================

\textbf{Pulvermacher Open Research License (PORL), Version 1.0}

Copyright~\copyright{} Raiko Pulvermacher

These materials (texts, images, concepts) may be used, shared, and adapted
under the following conditions:

\begin{enumerate}
  \item \textbf{Attribution} -- Any use, distribution, publication or
        citation must clearly and visibly credit the author:
        Raiko Pulvermacher.
  \item \textbf{Transparency of Derivatives} -- If the material is
        modified, extended or embedded into other contexts, it must be
        clearly stated that it is a derived work.
  \item \textbf{Research \& Publications} -- If this work is used as a
        basis for scientific research, experiments, simulations,
        publications, preprints or talks, the author must be informed
        prior to or at the latest upon publication.
  \item \textbf{Disclaimer} -- This work does not represent a physical
        theory, formal model, or verified scientific claim. It is
        provided without warranty.
\end{enumerate}

\bigskip
\noindent
\textbf{Contact:}\\
Raiko Pulvermacher\\
E-Mail: \href{mailto:Pulvermacher.Raiko@web.de}{Pulvermacher.Raiko@web.de}\\
ORCID: \href{https://orcid.org/0009-0003-9431-1001}{https://orcid.org/0009-0003-9431-1001}\\
OSF: \href{https://osf.io/py42t/}{https://osf.io/py42t/}

\end{document}
